\newpage
\section{Der Intel 4004}

Der 4004 ist der erste von Intel entwickelte Mikroprozessor und einer der Ersten überhaupt. Er gilt als einer der großen Meilensteine, die den Siegeszug der Computer einleitete. Schon Jahre vor seiner Entwicklung fingen Halbleiterchips an, die alten Elektronenröhren Rechner abzulösen. Durch die schnell voranschreitende Miniaturisierung von Transistoren ließen sich immer komplexere Logik Konstruktionen auf immer kleineren Chips umsetzen. Was vielversprechend begann, stellte sich allerdings bald als Problem heraus. Als die Komplexität und Spezialisierung einzelner Designs so stark zunahm, dass es nicht mehr kosteneffizient war manche Designs umzusetzen. Die Chips waren so spezialisiert geworden, dass die geringe Absatzmenge nicht die Entwicklungskosten rechtfertigte. Deshalb war es nur eine Frage der Zeit bis ein Universalrechner aus Silizium gebaut wurde.
 
\subsection{Die Geschichte und Entwicklung des 4004}

Als im Sommer 1969 Busicom, ein japanischer Hersteller von elektrischen Rechenmaschinen, die Firma Intel damit beauftragte die Chips für ihre neue Reihe von Rechenmaschinen zu produzieren, war Intel gerade mal ein Jahr alt. Das Unternehmen war eigentlich ein Hersteller von Halbleiter Speicherchips und hatte zum Zeitpunkt des Auftrags nur zwölf Mitarbeiter. Busicom verwendete 